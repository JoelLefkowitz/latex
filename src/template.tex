\documentclass[12pt]{article}
\usepackage{amsmath, amsthm}
\usepackage[utf8]{inputenc}
\usepackage[english]{babel}

\textwidth 12cm
\textheight 20cm

\newtheorem{theorem}{Theorem}[section]
\newtheorem{corollary}{Corollary}[theorem]
\newtheorem{lemma}[theorem]{Lemma}

\renewenvironment{abstract}
 {\small
  \begin{center}
  \bfseries \abstractname\vspace{-.5em}\vspace{0pt}
  \end{center}
  \list{}{
    \setlength{\leftmargin}{.5cm}%
    \setlength{\rightmargin}{\leftmargin}%
  }%
  \item\relax}
 {\endlist}

\begin{document}
\title{Existence and uniqueness of solutions for multidimensional nonlinear partial differential equations}
\author{Joel Reuben Lefkowitz \\
  \multicolumn{1}{p{.9\textwidth}}{\centering\emph{Undergraduate at The Univerity of Cambridge, Department of Engineering}}}
\maketitle

\begin{abstract}
  Abstract.
\end{abstract}
\section{Introduction}
Text.

\begin{theorem}
  Text.
\end{theorem}

\begin{theorem}[Pythagorean theorem]
  \label{pythagorean}
  \[ x^2 + y^2 = z^2 \]
\end{theorem}

A consequence of theorem \ref{pythagorean}:

\begin{corollary}
  Text.
\end{corollary}

\begin{lemma}
  Text.
\end{lemma}

\begin{thebibliography}{99}

  \bibitem{gasrah} G. Gasper, M. Rahman, {\it Basic Hypergeometric Series}, Cambridge University Press, Cambridge (1990).

\end{thebibliography}
\end{document}
